
\title{Regular sequences Compendium}

\begin{document}

\section{Definitions}
Let $q \geq 2$ be a fixed integer and $x$ be a sequence on $\mathbb{Z}_{\geq 0}{ }^{3}$ Then $x$ is said to be $(\mathbb{C}, q)$-regular (briefly: $q$-regular or simply regular) if the $\mathbb{C}$-vector space generated by its $q$-kernel

$$
\left\{x \circ\left(n \mapsto q^{j} n+r\right): \text { integers } j \geq 0,0 \leq r<q^{j}\right\}
$$

has finite dimension. In other words, $x$ is $q$-regular if there are an integer $D$ and sequences $x_{1}, \ldots, x_{D}$ such that for every $j \geq 0$ and $0 \leq r<q^{j}$ there exist complex numbers $c_{1}, \ldots, c_{D}$ with

$$
x\left(q^{j} n+r\right)=c_{1} x_{1}(n)+\cdots+c_{D} x_{D}(n) \quad \text { for all } n \geq 0 .
$$
\\

By Allouche and Shallit the sequence $x$ is $q$-regular if and only if there exists a vector-valued sequence $v$ whose first component coincides with $x$ and there exist square matrices $A_{0}, \ldots, A_{q-1}$ such that

$$
v(q n+r)=A_{r} v(n) \quad \text { for } 0 \leq r<q \text { and } n \geq 0 .
$$

This is called a $q$-linear representation of the sequence $x$.
\\

One characterisation of a $q$-regular sequence is as follows: the sequence $x(n)$ is said to be $q$-regular if there are square matrices $A_{0}, \ldots, A_{q-1}$ and a vector-valued sequence $v(n)$ such that

$$
v(q n+r)=A_{r} v(n) \quad \text { for } 0 \leq r<q \text { and } n \geq 0
$$

and such that $x(n)$ is the first component of $v(n)$.
\\

In the standard literature these sequences are called $k$-regular sequences (instead of $q$-regular sequences).
\\

Regular sequences are intimately related to the $q$-ary expansion of their arguments.
\\

Remark: Let $r_{\ell-1} \ldots r_{0}$ be the $q$-ary digit expansion ${ }^{4}$ of $n$. Then

$$
x(n)=e_{1} A_{r_{0}} \cdots A_{r_{\ell-1}} v(0)
$$

where $e_{1}=\left(\begin{array}{llll}1 & 0 & \ldots & 0\end{array}\right)$

Whenever we write that $r_{\ell-1} \ldots r_{0}$ is the $q$-ary digit expansion of $n$, we mean that $r_{j} \in\{0, \ldots, q-1\}$ for $0 \leq j<\ell, r_{\ell-1} \neq 0$ and $n=\sum_{0 \leq j<\ell} r_{j} q^{j}$. In particular, the $q$-ary expansion of zero is the empty Word.

\newpage
\section{Results}

We are interested in the asymptotic behaviour of the summatory function $X(N)=$ $\sum_{0 \leq n<N} x(n)$.
\\

Set $C:=\sum_{0 \leq r<q} A_{r}$. We choose $R>0$ such that $\left\|A_{r_{1}} \cdots A_{r_{\ell}}\right\|=O\left(R^{\ell}\right)$ holds for all $\ell \geq 0$ and $r_{1}, \ldots, r_{\ell} \in\{0, \ldots, q-1\}$. In other words, $R$ is an upper bound for the joint spectral radius of $A_{0}, \ldots, A_{q-1}$. The spectrum of $C$, i.e., the set of eigenvalues of $C$, is denoted by $\sigma(C)$. For $\lambda \in \mathbb{C}$, let $m(\lambda)$ denote the size of the largest Jordan block of $C$ associated with $\lambda$; in particular, $m(\lambda)=0$ if $\lambda \notin \sigma(C)$. 
\\

The scalar-valued Dirichlet series $\mathcal{X}$ and the vector-valued Dirichlet series $\mathcal{V}$ 

$$
\mathcal{X}(s)=\sum_{n \geq 1} n^{-s} x(n) \quad \text { and } \quad \mathcal{V}(s)=\sum_{n \geq 1} n^{-s} v(n)
$$

where $v(n)$ is the vector-valued sequence defined previously.
\\

With the notations above, we have Theorem A:

$$
\begin{aligned}
	X(N)= & \sum_{\substack{\lambda \in \sigma(C) \\
			|\lambda|>R}} N^{\log _{q} \lambda} \sum_{0 \leq k<m(\lambda)} \frac{(\log N)^{k}}{k !} \Phi_{\lambda k}\left(\left\{\log _{q} N\right\}\right) \\
	& +O\left(N^{\log _{q} R}(\log N)^{\max \{m(\lambda):|\lambda|=R\}}\right)
\end{aligned}
$$

for suitable 1-periodic continuous functions $\Phi_{\lambda k}$. If there are no eigenvalues $\lambda \in \sigma(C)$ with $|\lambda| \leq R$, the $O$-term can be omitted.
\\

$\mathcal{V}(s)$ satisfies the functional equation

$$
\left(I-q^{-s} C\right) \mathcal{V}(s)=\sum_{1 \leq n<q} n^{-s} v(n)+q^{-s} \sum_{0 \leq r<q} A_{r} \sum_{k \geq 1}\left(\begin{array}{c}
	-s \\
	k
\end{array}\right)\left(\frac{r}{q}\right)^{k} \mathcal{V}(s+k)
$$

for $\Re s>\log _{q} R$.
\\

$\mathcal{V}(s)$ can only have poles where $q^{s} \in \sigma(C)$
\\

The Fourier series

$$
\Phi_{\lambda k}(u)=\sum_{\ell \in \mathbb{Z}} \varphi_{\lambda k \ell} \exp (2 \ell \pi i u)
$$

converges pointwise for $u \in \mathbb{R}$ where the Fourier coefficients $\varphi_{\lambda k \ell}$ are defined by the singular expansion

$$
\frac{x(0)+\mathcal{X}(s)}{s} \asymp \sum_{\substack{\lambda \in \sigma(C) \\|\lambda|>R}} \sum_{\ell \in \mathbb{Z}} \sum_{0 \leq k<m(\lambda)} \frac{\varphi_{\lambda k \ell}}{\left(s-\log _{q} \lambda-\frac{2 \ell \pi i}{\log q}\right)^{k+1}}
$$
\\

Explicit formulation to algorithmically compute $\varphi_{\lambda k \ell}$

$$
\varphi_{\lambda k \ell}=\operatorname{Res}\left(\frac{x(0)+\mathcal{X}(s)}{s}\left(s-\log _{q} \lambda-\frac{2 \ell \pi i}{\log q}\right)^{k}, s=\log _{q} \lambda+\frac{2 \ell \pi i}{\log q}\right)
$$
\\
 
description of the Fourier coefficients of the $\Phi_{k_{j}}$

\section{Application on the sequence itself}

Let $x(N)$ be a $q$-regular sequence. We may rewrite it as a telescoping sum

$$
x(N)=x(0)+\sum_{n<N}(x(n+1)-x(n)) .
$$

The sequence of differences $x(n+1)-x(n)$ is again $q$-regular. Conversely, it is also well-known that the summatory function of a $q$-regular sequence is itself $q$-regular.

Therefore, we might also start to analyse a regular sequence by considering it to be the summatory function of its sequence of differences. In this way, we can apply all of the machinery developed in this article.
 
 

\section{Examples}

The best-known example for a 2-regular function is the binary sum-of-digits function.

Example: For $n \geq 0$, let $x(n)=s(n)$ be the binary sum-of-digits of $n$. We clearly have

$$
\begin{aligned}
	x(2 n) & =x(n), \\
	x(2 n+1) & =x(n)+1
\end{aligned}
$$

for $n \geq 0$. Indeed, we have

$$
x\left(2^{j} n+r\right)=x(n)+x(r) \cdot 1
$$

for integers $j \geq 0,0 \leq r<2^{j}$ and $n \geq 0$; i.e., the complex vector space generated by the 2-kernel is generated by $x$ and the constant sequence $n \mapsto 1$.

Alternatively, we set $v=(x, n \mapsto 1)^{\top}$ and have

$$
\begin{aligned}
	v(2 n) & =\left(\begin{array}{c}
		x(n) \\
		1
	\end{array}\right)=\left(\begin{array}{ll}
		1 & 0 \\
		0 & 1
	\end{array}\right) v(n) \\
	v(2 n+1) & =\left(\begin{array}{c}
		x(n)+1 \\
		1
	\end{array}\right)=\left(\begin{array}{ll}
		1 & 1 \\
		0 & 1
	\end{array}\right) v(n)
\end{aligned}
$$

for $n \geq 0$. Thus $(3.1)$ holds with

$$
A_{0}=\left(\begin{array}{ll}
	1 & 0 \\
	0 & 1
\end{array}\right), \quad A_{1}=\left(\begin{array}{ll}
	1 & 1 \\
	0 & 1
\end{array}\right)
$$

At this point, we note that a linear representation immediately leads to an explicit expression for $x(n)$ by induction.
\\

We have $C=A_{0}+A_{1}=\left(\begin{array}{ll}2 & 1 \\ 0 & 2\end{array}\right)$. As $A_{0}$ is the identity matrix, any product $A_{r_{1}} \cdots A_{r_{\ell}}$ has the shape $A_{1}^{k}=\left(\begin{array}{ll}1 & k \\ 0 & 1\end{array}\right)$ where $k$ is the number of factors $A_{1}$ in the product. This implies that $R$ with $\left\|A_{r_{1}} \cdots A_{r_{\ell}}\right\|=O\left(R^{\ell}\right)$ may be chosen to be any number greater than 1. As $C$ is a Jordan block itself, we simply read off that the only eigenvalue of $C$ is $\lambda=2$ with $m(2)=2$.
\\

Theorem A yields

$$
X(N)=N(\log N) \Phi_{21}\left(\left\{\log _{2} N\right\}\right)+N \Phi_{20}\left(\left\{\log _{2} N\right\}\right)
$$

\end{document}
